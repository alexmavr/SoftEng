\documentclass[a4paper,10pt]{article} \usepackage{anysize}
\marginsize{2cm}{2cm}{1cm}{1cm}
%\textwidth 6.0in \textheight = 664pt
\usepackage{xltxtra}
\usepackage{xunicode}
\usepackage{graphicx}
\usepackage{color}
\usepackage[table]{xcolor}
%\usepackage[usenames,dvipsnames]{xcolor}
%\usepackage{xgreek}
\usepackage{fancyvrb}
\usepackage{minted}
\usepackage{listings}
\usepackage{enumitem}
\usepackage{framed}
\usepackage{relsize}
\usepackage{float} 
\setmainfont[Mapping=TeX-text]{CMU Serif}

\definecolor{bootred}{RGB}{248,148,6}
\definecolor{bootgreen}{RGB}{81,163,81}
\definecolor{bootblue}{RGB}{73, 178, 205}

% Finally, give us PDF bookmarks                                                   
\usepackage{color,hyperref}                                                        
\definecolor{darkblue}{rgb}{0.0,0.0,0.3}                                           
\hypersetup{colorlinks,breaklinks,                                                 
    linkcolor=darkblue,urlcolor=darkblue,                                          
    anchorcolor=darkblue,citecolor=darkblue}                                       


\begin{document}

\begin{titlepage}
\begin{center}
\begin{figure}[t] 
     \includegraphics[scale=0.7]{title/ntua_logo}
\end{figure}
\begin{LARGE}\textbf{National Technical Univercity of Athens\\}\end{LARGE}
\vspace{2cm}
\begin{Large}
School of Electrical and Computer Engineering\\
Software Engineering Project Proposal\\
Project Title: Spyglass\\
Academic Year 2012-2013\\
\end{Large}
\vspace{6cm}
\begin{tabular}{l r}
\Large{Alex Maurogiannis}&
\large{(03109677)}\\
\Large{Gregory Lyras}&
\large{(03109687)}\\
\end{tabular}\\
\vspace{5cm}

\vfill
\large\today\\
\end{center}
\end{titlepage}



\tableofcontents

\pagebreak

\large{Document Sign-Off}

\begin{table}[h]
    \begin{tabular}{| c | c | c | c |}
        \hline
        Name & AM... & Signature & Date \\
        \hline
        Alex Maurogiannis & 03109677 & & \today \\
        \hline
        Gregory Lyras & 03109687 & & \today \\
        \hline
    \end{tabular}
\end{table}

\chapter{Introduction}

    \section{Purpose}
        The purpose of this document is to describe the design specifications
        of the \emph{spyglass} open source project.
    \section{Summary}
        This project aims to address the issue of dynamic content monitoring
        in a novel manner. Instead of searching on one site, we can monitor
        sites of interest and filter their updates based on keywords provided.
        If the updates match the search criteria then the end user is notified
        for the latest developments. The proposed project is a meta-search
        engine used to monitor sites for required information and provide
        notifications to its users accordingly. 
    \section{References}
        \begin{itemize}
                \item \href{https://www.djangoproject.com/}{django}
                \item \href{http://tastypieapi.org/}{tastypie}
                \item
                    \href{https://en.wikipedia.org/wiki/Representational_state_transfer}{REST framework}
                \item \href{https://github.com/mastergreg/spyglass-crawlie}{spyglass-crawlie}
                \item \href{https://github.com/afein/django-spyglass}{django-spyglass}
        \end{itemize}

\chapter{Design Choices}
    \section{Why Python}
        Python is a modern programming language. It features a dynamic type
        system and a strong object oriented model. It emphasises on code
        readability and less lines of code. One of its strongest features is
        'batteries included' and easy interfacing with C libraries. This
        provides a large ecosystem of available modules and libraries to
        choose from. When facing the task of such complexity as a generic web
        application, a programming language with the above features seems a
        good choice.
    \section{Why Django}
        The \emph{django} framework is a highly extensible modern web
        framework with a plethora of ready-to-use capabilities such as
        object-relation modelling, session management, dynamic templating and
        multiple \emph{DBMS} support that simplify the development of web
        applications. Instead of developing a web application from scratch we
        decided to utilise \emph{django's} features to focus on the task at
        hand.
    \section{Why REST}
        Given the recent advances in the standardisation of web Application
        Programming Interfaces (APIs) the use of RESTful architectural methods
        is highly preferable in contradiction to deprecated remote procedure
        call architectures such as SOAP. Additionally, there are several
        modules that allow django to provide RESTful APIs.
    \section{Crawler Network}
        The main design concept behind this project is to provide realtime
        monitoring of multiple resources. In order to provide this feature we
        designed a distributed method of crawling that provides the network
        with resilience against blocking. Thus we distribute the set of
        crawlers on several machines that operate independently and
        concurrently.
    \section{Crawler Protocol}
        The protocol used by the crawlers when they interact with the server
        provides the following operations. 
        \begin{itemize}
                \item \emph{Get Sites}: The crawler requests through HTTP a
                    list of websites supported by the server.
                \item \emph{Get XPaths}: The crawler requests through HTTP the
                    XPaths for every field of interest for each crawlable
                    website.
                \item \emph{Get Workload}: The crawler requests through HTTP a
                    list of queries to be resolved.
                \item \emph{Send Result}: If a query is resolved the crawler
                    sends through HTTP the data back to the server.
                \item \emph{Update Timestamps}: When the query is not resolved
                    the crawler sends a PATCH request to the server regarding
                    the unresolved query in order to update the query's
                    timestamp.
        \end{itemize}
\chapter{System Architecture}
\inputminted{text}{files/tags}
\chapter{Class Details}
\chapter{State Diagram}
\chapter{Open Subjects}
\end{document}
